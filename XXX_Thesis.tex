% Hauptdatei/ Steuerdatei f�r das gesamte Projekt
% Nehmen Sie bitte nur �nderungen an gekennzeichneten Stellen vor.
% Diese Stellen sind mit **manual** kenntlich gemacht.

% Klassenaufruf:
%
% Innerhalb der eckigen Klammern k�nnen Sie durch Komma getrennt
% nachfolgend aufgef�hrte Optionen in der Klasse steuern.
% Falls keine Optionen angegeben werden, verwendet der Compiler
% die mit * gekennzeichneten als Standard. Die Bedeutung entnehmen
% Sie bitte dem aus diesem Projekt generierten PDF Dokument.
%
% OPTIONEN:
% oneside* <-> twoside
% onecolumn* <-> twocolumn
% final* <-> draft
% fleqn
% rgbt* <-> flbt
% en <-> de*
% pdf* <-> nonpdf
%
% **manual**
\documentclass[oneside, onecolumn, final, fleqn, rgbt, en, pdf]{ESM_Latex2e_style}
%\usepackage[english]{babel}
\usepackage{pdfpages}
\usepackage{graphicx}
\usepackage{blindtext}
\usepackage{listings}
\usepackage{color}

\definecolor{dkgreen}{rgb}{0,0.6,0}
\definecolor{gray}{rgb}{0.5,0.5,0.5}
\definecolor{mauve}{rgb}{0.58,0,0.82}

\lstset{frame=tb,
	language=Java,
	aboveskip=3mm,
	belowskip=3mm,
	showstringspaces=false,
	columns=flexible,
	basicstyle={\small\ttfamily},
	numbers=none,
	numberstyle=\tiny\color{gray},
	keywordstyle=\color{blue},
	commentstyle=\color{dkgreen},
	stringstyle=\color{mauve},
	breaklines=true,
	breakatwhitespace=true,
	tabsize=3
}

% Integrated Packages
%
% [latin1]{inputenc}	- German Keyboard
% [T1]{fontenc}
% {graphicx}			- Graphics	
% {epstopdf}			- for the usage of eps-grafics
% {times}
% {makeidx}				- automatic index
% {color}
% {supertabular}      	- tables over multiple pages
% {rotating}			- rotation of objects and sidewaystables
% {float}				- fixation of floats
% {textcase}			- textformating, e.g. transformations


% Optionale Usepackages
%
% Die folgenden Packages wurden hinzugef�gt,
% da diese oft Verwendung finden. Diese k�nnen jedoch
% mit einem Kommentarzeichen ausgeschaltet werden, da
% User diese u.U. nicht ben�tigen.
%
% **manual**
% \usepackage{layout}             % graphische Darstellung der Abmessungen im Kapitel Abmessungen
                                % das usepackage layout kann f�r die wissenschaftliche Arbeit
                                % entfernt werden

%\usepackage{booktabs}           % Besondere Tabellenumgebung wie man sie oft im modernen
                                % Buchdruck findet
\newcommand*\chem[1]{\ensuremath{\mathrm{#1}}}
%   \usepackage{flafter}        % flafter = float after -> Kein Gleitobjekt vor Einbindungsstelle
%
%   \usepackage{amsmath}        % Erleichter die verbesserte Ausgabe bei der Verwendung mathematischer
%                               %  Formulierungen in Texten; z.B. ist split eine sch�ne Hilfe zur Darstellung
%                               % mehrzeiliger Formeln ohne eqnarray.

%   \usepackage{longtable}      % Tabellen k�nnen sich �ber Seiten fortsetzen, ohne besonderes
%                               % Einwirken durch den Benutzer, �hnlich wie supertabular.
%                               % Supertabular hat jedoch das umfangreichere Funktionsspektrum.

%
%
% User Usepackages
%
% F�gen Sie nachfolgend von Ihnen ben�tigte Usepackes ein.
% \usepackage[option]{packagename}
%
% **manual**
%
%
% \usepackage{pdfpages}						% F�r das Einbinden von mehrseitige pdf-Dateien

% \usepackage{rotating}						% F�r das Rotieren von Tabellen

\usepackage[bf, it, textfont=it, hang]{caption}		% Um das 'Caption' zu personalisieren

\makeindex
% 


%\usepackage{subcaption}

%\usepackage{placeins}
%\usepackage[hidelinks]{hyperref} % crosslinks inside the document
%\usepackage{url}

%\usepackage[table]{xcolor}
%\usepackage{xspace}


%----------------------------------------------------------------
% DOKUMENT ANFANG
\begin{document}
\pagenumbering{Alph}

%  TITELSEITE EINBINDEN
% +++++++++++++++++++++++++++++++++++++++++++++++

%
% ########################################################
% TITEL-SEITE
%
% =======================================================
% (1) W�hlen Sie die Art Ihrer Abschlu�arbeit: nichtzutreffendes auskommentieren
% (2) Setzen Sie in \title den Titel Ihrer Abschlu�arbeit ein.
% (3) Ersetzen Sie " Autor durch Ihren Namen"
% (4) Setzen Sie das gew�nschte Datum, oder verwenden das auto - Datum
% (5) Geben Sie Ihre(n) Hochschulbetreuer an.
% =======================================================
%
%   (1)

\Bachelor		% Masterarbeit
%\Bachelor	% Bachelorarbeit
%\Diplom	% Diplomarbeit
%\Stud		% Studienarbeit
%\Hsem		% Hauptseminar
%\Ppra		% Projektpraktikum
%\Forpraxis	% Forschungspraxis
%\Ingpraxis	% Ingenieurspraxis

%   (2)
\title{Designing a Circuit for Alphasense NO2-B43F Sensor}

%   (3)
%\author{Max Mustermann}
\author{Emre �zbas}

%   (4)
%\date{\scalebox{.9}[.95]{26. September 2016}}
\date{\today}

%   (5)
\Betreuer{Prof. Dr.- Ing. Jia Chen}
%
% =======================================================
% Die Verwendung von thanks wird durch die Klasse unterbunden.
% Verwenden Sie stattdessen \textbf{} die Vorwort-Umgebung
% =======================================================
%
\maketitle  % !! NICHT VER�NDERN !!
% =======================================================
% Der ESM Kopf erscheint nicht im DVI!!
% Es ist eine �bersetzung nach pdf notwendig!
% =======================================================
% ENDE TITEL-SEITE
% ########################################################

\pagenumbering{Roman}
%
% ZUSAMMENFASSUNG EINBINDEN
% +++++++++++++++++++++++++++++++++++++++++++++++
% Kurzfassung in Deutsch oder Englisch, maximal 650 Zeichen inkl. Leerzeichen
\chapter*{Abstract}

Scientific data shows that living in locations with high density of nitrogen dioxide can cause severe damage to human health and result in a lower quality of life. The researches indicate that main reason for \chem{NO_2} emission is mainly caused by burning fossil fuels, industrial processes and especially by motor vehicle exhaust which constitutes approximately 80\% of the total \chem{NO_2} emission in cities. Therefore it is important to track the air pollutant emission to prevent any harm to human health as well as financial damage and its consequences.\par 
In this bachelor thesis it is aimed to design and build a gas measurement sensor unit which tracks and saves voltage data linearly proportional to air pollutant density. The measurement unit is designed for Alphasense 4-electrode B-type sensors and is tested with Alphasense NO2-B43F sensor. The hardware section of the circuit is explained in detail as well as the software segment is documented to enable a faster approach to this matter and realization with minimal problems. The results obtained from the sensor unit elicit that it is possible to track and save data concerning \chem{NO_2} measurement within a reasonable margin of error, using minimal resources.     


\newpage
%
%
% VORWORT: F�R VORHERGEHENDE ERL�UTERUNGEN UND EVTL. DANKWORTE
% +++++++++++++++++++++++++++++++++++++++++++++++
% Ab hier ist die Umgebung optional und dem entsprechenden Bed�rfnis angepa�t einzusetzen!
% Falls die Umgebung nicht ben�tigt wird, entweder mit einem Kommentarzeichen deaktivieren,
% oder den Input l�schen.
%
% **manual**
%\input{TeX/vorwort}
%\input{TeX/acknowledgement}
\vspace*{2\majorheadskip}
I confirm that this Master's Thesis is my own work and I have documented all sources and material used.

\vspace{3cm}

{\def\arraystretch{.7}
\hspace{.75cm}
\begin{tabular}{ccc}
	Munich, \today		&	\hspace{3cm}	& \\
	\rule{4.5cm}{0.4pt}					&					& \rule{4.5cm}{0.4pt}\\
	Place, Date							&					& Signature
\end{tabular}
}



\clearpage
%
%
% INHALTSVERZEICHNIS
% +++++++++++++++++++++++++++++++++++++++++++++++
\tableofcontents
%
%
% EINLEITUNG EINBINDEN
% +++++++++++++++++++++++++++++++++++++++++++++++
%\setcounter{page}{1}%
\chapter{Introduction}
\label{sec:introduction}


As the world population increases by millions every year, the environmental damage we cause increases dramatically. In densely populated areas -and especially in larger cities- air pollution is a major problem, which does not only have financial consequences but also affects the quality of our lives in many ways. Since air pollution today constitutes a significant problem, there are more than many researches and studies regarding this issue: " . . . the total damage costs of air pollution [is estimated] to be US\$ 3.0 trillion in 2010, or 5.6\% of Gross World Product (GWP). These losses are equivalent to US\$ 430 for every person on the planet."\cite{Hutton2011} is from just one of the numerous studies made on financial damage caused by air pollution.\par
To be able to assess this problem correctly and take suitable measures to minimize the harm of air pollution, one should first be capable of finding out the cause accurately. Only after an accurate diagnosis can there be a suitable solution and thus a significant outcome. When it comes to air pollution, the best way to detect the cause is to make density measurements of air pollutants with electrochemical sensors sensitive to specific gases in various locations. However there are some requirements that must be fulfilled: "To adequately characterize air quality (AQ), measurements must be fast (real-time), scalable, and reliable (with known accuracy, precision, and stability over time)."\cite{Cross2017} The more accurate and fast the sensors get, the more expensive the gas measurement station will be. Since it is important to make measurements in multiple locations to create a pool of air pollutant density data and thus getting a better understanding of the environmental damage, a collective of stations are needed, which increases the total cost dramatically.\par





asdasd%tezin gidişatı, aşamalar

%
%
% HAUPTTEIL EINBINDEN
% +++++++++++++++++++++++++++++++++++++++++++++++
% Die Eingebundene Datei Hauptteil sollte erhalten bleiben.
% Zum Einen k�nnen Sie dann den kompletten Hauptteil
% auf einmal auskommentieren, wenn Sie an anderen
% Teilen arbeiten. Ausserdem erh�ht die Einbindung Ihrer
% Kapitel in der Datei Hauptteil.tex die �bersichtlichkeit.
% Zudem finden sich die einzelnen Einbingsstellen gut, wodurch
% sich auch einzelne Kapitel auskommentieren lassen.
%
% HIER SIND DIE KAPITEL DER ARBEIT IN GEEIGNETER FORM EINZUBINDEN

\chapter{First Steps}
\label{sec:firststeps}

\section{Research}
After deciding to work on this subject, some research on the working principle of Alphasense \chem{NO_2} electrochemical sensor had to be made prior to the initialization of the mechanical and electronical sections of this project. A paper on previous experiments conducted in Boston, United States of America was very helpful to get a first idea about how I could start building my circuit designed for \chem{NO_2} density measurements. 
\chapter{Challenges}
\label{sec:challenges}

\section{Converting SMD to THT}
After preparing an orders list according to the previously mentioned circuit design taken from Alphasense \cite{2009} and receiving these components, my first challenge was very clear: Transforming the surface-mount device (SMD) parts into through-hole components. The operational amplifiers used in the circuit (LT6011) were only available to order in SMD form. Since I needed the SMD components to verify the circuit design, I could not simply design a printed circuit board and solder the components on the board without testing the circuit design with a breadboard first. Otherwise, everything would be inalterable and the tiniest change in circuit schematic would lead to a whole new circuit design and thus a printed circuit board from scratch. That is why I started to build an adapter for making the SMD modules breadboard compatible. \par 
For this purpose, I used a perfboard and male sockets. I first split the male sockets into two 4-pin male sockets, since the LT6011 ICs have an 8-pin structure, namely 4 pins on both sides. Afterward, I soldered these sockets parallel to each other with a reasonable distance in between so that there is enough room for the IC to fit between the two sockets. Here it is quite important to first solder the sockets on the side of the perfboard with copper rings around the holes with the longer part of the sockets laying on this side. The IC will be fixed on the other side of the perfboard because the distance between any two pins of the IC is much smaller than the distance between two holes of the perfboard and this would thus lead to a short between the pins. Afterward, I placed the IC between the sockets and anchored it on the perfboard by gluing it with hot glue. This makes it easier to make the connections between the sockets and the IC itself since it stays fixed during soldering. It can be tricky to solder one pin of the IC with one pin of the socket since there is no such surface e.g copper between those two on which the solder can stick. This makes the soldering process much harder since the solder tends to stay on one the IC or the socket, thus not binding them together. It is therefore recommended to make the distance between the two sockets as small as possible so that the IC and the socket stay as close as possible. This makes the soldering much easier since the distance between the two parts is smaller and a connection between the two can be made with enough solder applied. Another option would be to use tiny wires between the parts. This way the solder can stick onto this wire and make the connection instead of just creating a solder bubble on one pin. Here it is recommended to fix the wire by gluing it on the perfboard or use a third-hand soldering stand to stabilize the wire and the other parts.

\section{Initial Design}

\subsection{Inconsistencies Between Design and Sensor}
The circuit schematic taken from Alphasense webpage \cite{2009} illustrates a design for a 3-electrode sensor, but what we wanted to build was a circuit for the 4-electrode sensor NO2-B43F. This led to some problems concerning the correctness of the output data. The three electrodes of the sensor in the schematic, namely the working, counter and reference electrodes, have the same purposes like the three electrodes of the NO2-B43F. However, according to this design, the auxiliary electrode was not used at all, which is the fourth electrode of our sensor. This electrode is used for noise cancellation and zero drift correction. Not using the auxiliary electrode led to an output with high noise ratio, because firstly the electrode was floating (not connected to any stable potential) and secondly it could not be used for corrections in the output voltage. 

\subsubsection{Testing the Sensors}
Since we did not have any containers of \chem{NO_2} in the lab, I had to find another way to test the response of both sensors (the one on the ISB circuit and the one connected to my circuit) to a change of \chem{NO_2} concentration in the air. So I fixed the sensors in close proximity and parallel to each other onto a portable board in order to have the same environmental conditions for both of the sensors. Afterward, I fanned the sensors with the sensing areas of the sensors facing directly to the fan. Normally gas flow parallel to the sensing area of the sensor must be supplied to the sensors, since the sensors are sensitive to air flow perpendicular to the sensing area which causes a change in the outputted signal although the \chem{NO_2} concentration in the air does not change. For this reason, the sensors were fanned in order to create a perpendicular gas flow and thus create a change in the output signals of both sensors, of course only for testing purposes. This way I could compare the reaction velocity and amplitude of the output signals in millivolts.\par 
For the reasons explained in the previous section the outputted voltage values were different compared to the values gathered from the ISB circuit. The two circuit boards were tested in the same environment and for both circuits, the same power supply unit was used. For that reason, the values should have been close to each other. I tried to solve this problem by changing the software and adapting it to suppress the noise consisted of the output signal. For this reason, I added another function to the Arduino code in order to make the Arduino print the average voltage level over a specific time interval instead of printing the values directly. This resulted in better values with lower SNR, but the response velocity and amplitude were different compared to the values from the ISB circuit. Additionally, this was not a befitting solution since the values coming from my circuit were faulty and a correction in the hardware was needed.  



\section{Final Design}
\subsection{Critical parameters}
\label{sub:criticalParameters}

It is of utmost importance to pay attention to the resistor values in the circuit schematic and the resistors used in the actual circuit. A little difference in these values creates a large spread between the desired output voltage and the actual output since most of the resistors in this circuit design play a role in determining the gain of the operational amplifiers. An inaccuracy in resistor value changes the gain of the operational amplifier and consequently the output voltage gets multiplied with that error. If the resistor inaccuracy is in earlier stages of the circuit, the error at the output side gets even higher since the error gets amplified by the second operational amplifier in the last amplifying stage. It is recommended to check the resistors' values preferably with a multimeter before mounting it on the circuit, even if the circuit is being built on a breadboard and will be modifiable, since there are a lot of resistors in this circuit design and it can be quite complicated to detect where the problem lies after the circuit is completed. For this reason, it is quite important to check every part individually before putting them into the circuit. \par 

\subsection{Inconsistencies Between Design and Board}
After comparing the circuit design in Figure  ~\ref{fig:A.2} and the ISB board, I detected some discrepancies between the two. I found out that some of the resistors' values were different from each other. The values of the resistors $R29$ and $R31$ were depicted as 4.7 kilohms but the resistors on the ISB board were actually 1.5 kilohm resistors. This leads to a different input voltage at the non-inverting inputs of the ICs, which causes a different result outputted from the circuit. After correcting these values and editing out the parts which were not used in this project, I came up with the circuit schematic in Figure  ~\ref{fig:A.3}. 



\subsubsection{Testing the Sensors}
The results from my circuit were higher than the output values of the ISB circuit. For this reason, I firstly subtracted a constant number from the voltage values. As the outputs were stable, meaning that the sensors were not fanned and the test was conducted indoors with little to no change in the \chem{NO_2} concentration in the air, the values were approximately equal to each other. However after fanning the sensors the difference in the response amplitude and response velocity was noticeable. 
\chapter{Outlook}
\label{sec:outlook}
\section{Conclusion}

In this bachelor thesis, an initial design was tested and problems were detected and documented for facilitating future realizations. Afterward, the voltage data outputted from this sensor unit was saved and plotted for better visualization and documentation purposes. These values indicate an unacceptable margin of error compared to the ISB circuit from Alphasense since the auxiliary electrode was not being used in this sensor design and thus not protected against electrical and electromagnetic noise. Therefore a new circuit schematic was built and tested with the same electrochemical sensor. Since the auxiliary electrode was not left out in the later design, the circuit was a success and could supply similar results as the ISB board. Additionally with the help of low noise voltage regulators and the use of more resistor-capacitor pairs as noise filters the circuit was more resistant to electrical noise and resulted in an output signal with a higher signal-to-noise ratio.\par 
In conclusion, the last results were success in terms of being able to present similar results like the values from the ISB board. The constructed circuit can now be used for data collection regarding toxic gas concentration; and output values, which can then be converted to ppb values. The goal of this bachelor thesis is thus met in terms of successful results and reconstructability. 

\section{Possible Future Enhancements}
Directly measuring the output voltage gives us, in theory, the raw data of the \chem{NO_2} concentration, since the output voltage increases or decreases linearly as the ppb level of \chem{NO_2} changes. However, the data will depend on environmental elements -especially from humidity and temperature- and thus be somewhat inconsistent. The raw data obtained directly from the circuit must be calibrated against such noises in order to get a better approximation of real concentration values. This can be achieved by altering the circuit and making it more resistant to noise. There are two possible ways of dealing with such noises: Passive and active filtering. The former is achieved by adding passive elements like resistors, capacitors, and inductors. After detecting the peak frequency of the noise contained in the raw signal (possibly with the help of an oscilloscope by applying Fourier Transformation on the signal and acquiring the frequency spectrum as well as peak frequency of the noise content) it can be eliminated by adding high and low-pass filters. The latter option requires an outside power source, hence the name "active", which makes it a more expensive option than the former, however, this can amplify the desired frequency while suppressing the noise, which makes the signal-to-noise ratio greater. Both options can be useful and serve our purpose of noise cancellation well.\par 
Additionally, data calibration against humidity and temperature could be achieved by integrating a humidity and temperature sensor into the circuit board and measuring their effect on the output voltage and then subtracting these elements from the raw data itself. As a result, the circuit would be more resistant against humidity and temperature changes in the environment and the outputted values would be even more dependent on the \chem{NO_2} concentration rather than other environmental effects.\par
Lastly, it is necessary to test the circuit under different environmental conditions with different \chem{NO_2} concentrations. The method used for examining the sensors were fine for testing purposes, although another experiment with varying \chem{NO_2} concentrations can certify the success of the outcome.






%  ANHANG
% +++++++++++++++++++++++++++++++++++++++++++++++
% Der Anhang wird durch die Umgebung appendix vom
% Rest der Arbeit abgegrenzt und hat eine eigene
% Darstellung der �berschriften und Kopfzeilen.
% Binden Sie ihre Anhangskapitel in der Datei Anhang.tex ein.

\begin{appendix}
    \chapter{Some Appendix}
\label{sec:Appendix}

text text text text

\chapter{Source Code}
\label{sub:appendixA}

text text text text


\end{appendix}

% ENDE ANHANG
%
% FORMELZEICHEN UND EINHEITEN
% +++++++++++++++++++++++++++++++++++++++++++++++
%
%\def\chaptermark#1{}
%\mark{{}{}}   % Initialisierung der Markierungen (wirkt wie ein Reset)
%\input{TeX/List-of-Symbols}
%
%\input{TeX/List-of-Abbreviations}%
%
% VERZEICHNIS DER VORKOMMENDEN ABBILDUNGEN
% +++++++++++++++++++++++++++++++++++++++++++++++
%\listoffigures
%
%
% VERZEICHNIS DER VORKOMMENDEN TABELLEN
% +++++++++++++++++++++++++++++++++++++++++++++++
%\listoftables
%
%
% LITERATURVERZEICHNIS
% +++++++++++++++++++++++++++++++++++++++++++++++
% Wichtig: Dieses ist immer einspaltig ausgef�hrt.
%%bibliographystyle{acm}
%\bibliographystyle{plain}
\bibliographystyle{unsrt}
\nocite{*}	% show all entries in bib-file
\bibliography{Literature}
%
%
%
%
% INDEX
% +++++++++++++++++++++++++++++++++++++++++++++++
% Hierzu muss an entsprechender Stelle im Text mit
% \index{Bezeichnung} eine Referenz erzeugt werden
%
% **manual**
%\printindex
%
%

\end{document}
% DOKUMENT ENDE
% ----------------------------------------------------------------

