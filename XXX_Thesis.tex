% Hauptdatei/ Steuerdatei f�r das gesamte Projekt
% Nehmen Sie bitte nur �nderungen an gekennzeichneten Stellen vor.
% Diese Stellen sind mit **manual** kenntlich gemacht.

% Klassenaufruf:
%
% Innerhalb der eckigen Klammern k�nnen Sie durch Komma getrennt
% nachfolgend aufgef�hrte Optionen in der Klasse steuern.
% Falls keine Optionen angegeben werden, verwendet der Compiler
% die mit * gekennzeichneten als Standard. Die Bedeutung entnehmen
% Sie bitte dem aus diesem Projekt generierten PDF Dokument.
%
% OPTIONEN:
% oneside* <-> twoside
% onecolumn* <-> twocolumn
% final* <-> draft
% fleqn
% rgbt* <-> flbt
% en <-> de*
% pdf* <-> nonpdf
%
% **manual**
\documentclass[oneside, onecolumn, final, fleqn, rgbt, en, pdf]{ESM_Latex2e_style}
%\usepackage[english]{babel}

% Integrated Packages
%
% [latin1]{inputenc}	- German Keyboard
% [T1]{fontenc}
% {graphicx}			- Graphics	
% {epstopdf}			- for the usage of eps-grafics
% {times}
% {makeidx}				- automatic index
% {color}
% {supertabular}      	- tables over multiple pages
% {rotating}			- rotation of objects and sidewaystables
% {float}				- fixation of floats
% {textcase}			- textformating, e.g. transformations


% Optionale Usepackages
%
% Die folgenden Packages wurden hinzugef�gt,
% da diese oft Verwendung finden. Diese k�nnen jedoch
% mit einem Kommentarzeichen ausgeschaltet werden, da
% User diese u.U. nicht ben�tigen.
%
% **manual**
% \usepackage{layout}             % graphische Darstellung der Abmessungen im Kapitel Abmessungen
                                % das usepackage layout kann f�r die wissenschaftliche Arbeit
                                % entfernt werden

%\usepackage{booktabs}           % Besondere Tabellenumgebung wie man sie oft im modernen
                                % Buchdruck findet


%   \usepackage{flafter}        % flafter = float after -> Kein Gleitobjekt vor Einbindungsstelle
%
%   \usepackage{amsmath}        % Erleichter die verbesserte Ausgabe bei der Verwendung mathematischer
%                               %  Formulierungen in Texten; z.B. ist split eine sch�ne Hilfe zur Darstellung
%                               % mehrzeiliger Formeln ohne eqnarray.

%   \usepackage{longtable}      % Tabellen k�nnen sich �ber Seiten fortsetzen, ohne besonderes
%                               % Einwirken durch den Benutzer, �hnlich wie supertabular.
%                               % Supertabular hat jedoch das umfangreichere Funktionsspektrum.

%
%
% User Usepackages
%
% F�gen Sie nachfolgend von Ihnen ben�tigte Usepackes ein.
% \usepackage[option]{packagename}
%
% **manual**
%
%
% \usepackage{pdfpages}						% F�r das Einbinden von mehrseitige pdf-Dateien

% \usepackage{rotating}						% F�r das Rotieren von Tabellen

\usepackage[bf, it, textfont=it, hang]{caption}		% Um das 'Caption' zu personalisieren

\makeindex
% 


%\usepackage{subcaption}

%\usepackage{placeins}
%\usepackage[hidelinks]{hyperref} % crosslinks inside the document
%\usepackage{url}

%\usepackage[table]{xcolor}
%\usepackage{xspace}


%----------------------------------------------------------------
% DOKUMENT ANFANG
\begin{document}
\pagenumbering{Alph}

%  TITELSEITE EINBINDEN
% +++++++++++++++++++++++++++++++++++++++++++++++

%
% ########################################################
% TITEL-SEITE
%
% =======================================================
% (1) W�hlen Sie die Art Ihrer Abschlu�arbeit: nichtzutreffendes auskommentieren
% (2) Setzen Sie in \title den Titel Ihrer Abschlu�arbeit ein.
% (3) Ersetzen Sie " Autor durch Ihren Namen"
% (4) Setzen Sie das gew�nschte Datum, oder verwenden das auto - Datum
% (5) Geben Sie Ihre(n) Hochschulbetreuer an.
% =======================================================
%
%   (1)

\Bachelor		% Masterarbeit
%\Bachelor	% Bachelorarbeit
%\Diplom	% Diplomarbeit
%\Stud		% Studienarbeit
%\Hsem		% Hauptseminar
%\Ppra		% Projektpraktikum
%\Forpraxis	% Forschungspraxis
%\Ingpraxis	% Ingenieurspraxis

%   (2)
\title{Building a Circuit for 4-Electrode Alphasense Sensors}

%   (3)
%\author{Max Mustermann}
\author{Emre �zbas}

%   (4)
%\date{\scalebox{.9}[.95]{26. September 2016}}
\date{\today}

%   (5)
\Betreuer{Prof. Dr.- Ing. Jia Chen}
%
% =======================================================
% Die Verwendung von thanks wird durch die Klasse unterbunden.
% Verwenden Sie stattdessen \textbf{} die Vorwort-Umgebung
% =======================================================
%
\maketitle  % !! NICHT VER�NDERN !!
% =======================================================
% Der ESM Kopf erscheint nicht im DVI!!
% Es ist eine �bersetzung nach pdf notwendig!
% =======================================================
% ENDE TITEL-SEITE
% ########################################################

\pagenumbering{Roman}
%
% ZUSAMMENFASSUNG EINBINDEN
% +++++++++++++++++++++++++++++++++++++++++++++++
% Kurzfassung in Deutsch oder Englisch, maximal 650 Zeichen inkl. Leerzeichen
\chapter*{Abstract}

text text text text


\newpage
%
%
% VORWORT: F�R VORHERGEHENDE ERL�UTERUNGEN UND EVTL. DANKWORTE
% +++++++++++++++++++++++++++++++++++++++++++++++
% Ab hier ist die Umgebung optional und dem entsprechenden Bed�rfnis angepa�t einzusetzen!
% Falls die Umgebung nicht ben�tigt wird, entweder mit einem Kommentarzeichen deaktivieren,
% oder den Input l�schen.
%
% **manual**
%\input{TeX/vorwort}
%\input{TeX/acknowledgement}
\vspace*{2\majorheadskip}
I confirm that this Master's Thesis is my own work and I have documented all sources and material used.

\vspace{3cm}

{\def\arraystretch{.7}
\hspace{.75cm}
\begin{tabular}{ccc}
	Munich, \today		&	\hspace{3cm}	& \\
	\rule{4.5cm}{0.4pt}					&					& \rule{4.5cm}{0.4pt}\\
	Place, Date							&					& Signature
\end{tabular}
}



\clearpage
%
%
% INHALTSVERZEICHNIS
% +++++++++++++++++++++++++++++++++++++++++++++++
\tableofcontents
%
%
% EINLEITUNG EINBINDEN
% +++++++++++++++++++++++++++++++++++++++++++++++
%\setcounter{page}{1}%
\chapter{Introduction}
\label{sec:introduction}


As the world population increases by millions every year, the environmental damage we cause increases dramatically. In densely populated areas -and especially in larger cities- air pollution is a major problem, which does not only have financial consequences but also affects the quality of our lives in many ways. Since air pollution today constitutes a significant problem, there are more than many researches and studies regarding this issue: " . . . the total damage costs of air pollution [is estimated] to be US\$ 3.0 trillion in 2010, or 5.6\% of Gross World Product (GWP). These losses are equivalent to US\$ 430 for every person on the planet."\cite{Hutton2011} is from just one of the numerous studies made on financial damage caused by air pollution.\par
To be able to assess this problem correctly and take suitable measures to minimize the harm of air pollution, one should first be capable of finding out the cause accurately. Only after an accurate diagnosis can there be a suitable solution and thus a significant outcome. When it comes to air pollution, the best way to detect the cause is to make density measurements of air pollutants with electrochemical sensors sensitive to specific gases in various locations. However there are some requirements that must be fulfilled: "To adequately characterize air quality (AQ), measurements must be fast (real-time), scalable, and reliable (with known accuracy, precision, and stability over time)."\cite{Cross2017} The more accurate and fast the sensors get, the more expensive the gas measurement station will be. Since it is important to make measurements in multiple locations to create a pool of air pollutant density data and thus getting a better understanding of the environmental damage, a collective of stations are needed, which increases the total cost dramatically.\par
The goal of this bachelor thesis is to design and build a low-cost circuit suited for Alphasense NO2-B43F sensor, which is sensitive to \chem{NO_2} concentration in the air.



%tezin gidişatı, aşamalar

%
%
% HAUPTTEIL EINBINDEN
% +++++++++++++++++++++++++++++++++++++++++++++++
% Die Eingebundene Datei Hauptteil sollte erhalten bleiben.
% Zum Einen k�nnen Sie dann den kompletten Hauptteil
% auf einmal auskommentieren, wenn Sie an anderen
% Teilen arbeiten. Ausserdem erh�ht die Einbindung Ihrer
% Kapitel in der Datei Hauptteil.tex die �bersichtlichkeit.
% Zudem finden sich die einzelnen Einbingsstellen gut, wodurch
% sich auch einzelne Kapitel auskommentieren lassen.
%
% HIER SIND DIE KAPITEL DER ARBEIT IN GEEIGNETER FORM EINZUBINDEN

\chapter{Challenges}
\label{sec:challenges}


\section{Initial Design}
\subsection{Mechanical and optical challenges}
\label{sub:mechanicalChallenges}

After preparing an orders list according to the previously mentioned circuit design taken from Alphasense \textbf{dokuman} and receiving these components, my first challenge was very clear: Transforming the surface-mount device (SMD) parts into through-hole components. The operational amplifiers used in the circuit (LT6011) were only available to order in SMD form. Since I needed the SMD components to verify the circuit design, I could not simply design a printed circuit board and solder the components on the board without testing the circuit design with a breadboard first. Otherwise everything would be inalterable and the tiniest change in circuit schematic would lead to a whole new circuit design and thus a printed circuit board from scratch. That is why I started to build an adapter for making the SMD modules breadboard compatible. \par 
For this purpose I used a perfboard and male sockets. I first split the male sockets into two 4-pin male sockets, since the LT6011 ICs have an 8-pin structure, namely 4 pins on both sides. Afterwards I soldered these sockets parallel to each other with reasonable distance in between so that there is enough room for the IC to fit between the two sockets. Here it is quite important to first solder the sockets on the side of the perfboard with copper rings around the holes with the longer part of the sockets laying on this side. The IC will be fixed on the other side of the perfboard because the distance between any two pins of the IC is much smaller than the distance between two holes of the perfboard and this would thus lead to a short between the pins. Afterwards I placed the IC between the sockets and anchored it on the perfboard by gluing it with hot glue. This makes it easier to make the connections between the sockets and the IC itself, since it stays fixed during soldering. It can be tricky to solder one pin of the IC with one pin of the socket since there is no such surface e.g copper between those two on which the solder can stick. This makes the soldering process much harder since the solder tends to stay on one the IC or the socket, thus not binding them together. It is therefore recommended to make the distance between the two sockets as small as possible, so that the IC and the socket stays as close as possible. This makes the soldering much easier since the distance between the two parts is smaller and a connection between the two can be made with enough solder applied. Another option would be to use tiny wires between the parts. This way the solder can stick onto this wire and make the connection instead of just creating a solder bubble on one pin. Here it is recommended to fix the wire by gluing it on the perfboard or use a third hand soldering stand to stabilize the wire and the other parts.     

\section{Final Design}
\subsection{Mechanical and optical challenges}


\subsection{Critical parameters}
\label{sub:criticalParameters}

It is of utmost importance to pay attention to the resistor values in the circuit schematic and the resistors used in the actual circuit. A little difference in these values creates a large spread between the desired output voltage and the actual output, since most of the resistors in this circuit design plays a role in determining the gain of the operational amplifiers. An inaccuracy in resistor value changes the gain of the operational amplifier and consequently the output voltage gets multiplied with that error. If the resistor inaccuracy is in earlier stages of the circuit, the error at the output side gets even higher since the error gets amplified by the second operational amplifier in the last amplifying stage. It is recommended to check the resistors' values preferably with a multimeter before mounting it on the circuit, even if the circuit is being built on a breadboard. There are a lot of resistors in this circuit design and it can be really complicated to detect where the problem lies after the circuit is completed. For this reason it is quite important to check every part individually before putting them into the circuit. \par 
In my case I built the circuit exatly how it was depicted on the circuit schematic and       
\chapter{Outlook}
\label{sec:outlook}

text text text text


%  ANHANG
% +++++++++++++++++++++++++++++++++++++++++++++++
% Der Anhang wird durch die Umgebung appendix vom
% Rest der Arbeit abgegrenzt und hat eine eigene
% Darstellung der �berschriften und Kopfzeilen.
% Binden Sie ihre Anhangskapitel in der Datei Anhang.tex ein.

\begin{appendix}
    \chapter{Some Appendix}
\label{sec:Appendix}

text text text text

\chapter{Source Code}
\label{sub:appendixA}

text text text text


\end{appendix}

% ENDE ANHANG
%
% FORMELZEICHEN UND EINHEITEN
% +++++++++++++++++++++++++++++++++++++++++++++++
%
%\def\chaptermark#1{}
%\mark{{}{}}   % Initialisierung der Markierungen (wirkt wie ein Reset)
%\input{TeX/List-of-Symbols}
%
%\input{TeX/List-of-Abbreviations}%
%
% VERZEICHNIS DER VORKOMMENDEN ABBILDUNGEN
% +++++++++++++++++++++++++++++++++++++++++++++++
%\listoffigures
%
%
% VERZEICHNIS DER VORKOMMENDEN TABELLEN
% +++++++++++++++++++++++++++++++++++++++++++++++
%\listoftables
%
%
% LITERATURVERZEICHNIS
% +++++++++++++++++++++++++++++++++++++++++++++++
% Wichtig: Dieses ist immer einspaltig ausgef�hrt.
%%bibliographystyle{acm}
%\bibliographystyle{plain}
\bibliographystyle{unsrt}
\nocite{*}	% show all entries in bib-file
\bibliography{Literature}
%
%
%
%
% INDEX
% +++++++++++++++++++++++++++++++++++++++++++++++
% Hierzu muss an entsprechender Stelle im Text mit
% \index{Bezeichnung} eine Referenz erzeugt werden
%
% **manual**
%\printindex
%
%

\end{document}
% DOKUMENT ENDE
% ----------------------------------------------------------------

