\chapter{Outlook}
\label{sec:outlook}
\section{Conclusion}

In this bachelor thesis, an initial design was tested and problems were detected and documented for facilitating future realizations. Afterward, the voltage data outputted from this sensor unit was saved and plotted for better visualization and documentation purposes. These values indicate an unacceptable margin of error compared to the ISB circuit from Alphasense since the auxiliary electrode was not being used in this sensor design and thus not protected against electrical and electromagnetic noise. Therefore a new circuit schematic was built and tested with the same electrochemical sensor. Since the auxiliary electrode was not left out in the later design, the circuit was a success and could supply similar results as the ISB board. Additionally with the help of low noise voltage regulators and the use of more resistor-capacitor pairs as noise filters the circuit was more resistant to electrical noise and resulted in an output signal with a higher signal-to-noise ratio.\par 
In conclusion, the last results were success in terms of being able to present similar results like the values from the ISB board. The constructed circuit can now be used for data collection regarding toxic gas concentration; and output values, which can then be converted to ppb values. The goal of this bachelor thesis is thus met in terms of successful results and reconstructability. 

\section{Possible Future Enhancements}
Directly measuring the output voltage gives us, in theory, the raw data of the \chem{NO_2} concentration, since the output voltage increases or decreases linearly as the ppb level of \chem{NO_2} changes. However, the data will depend on environmental elements -especially from humidity and temperature- and thus be somewhat inconsistent. The raw data obtained directly from the circuit must be calibrated against such noises in order to get a better approximation of real concentration values. This can be achieved by altering the circuit and making it more resistant to noise. There are two possible ways of dealing with such noises: Passive and active filtering. The former is achieved by adding passive elements like resistors, capacitors, and inductors. After detecting the peak frequency of the noise contained in the raw signal (possibly with the help of an oscilloscope by applying Fourier Transformation on the signal and acquiring the frequency spectrum as well as peak frequency of the noise content) it can be eliminated by adding high and low-pass filters. The latter option requires an outside power source, hence the name "active", which makes it a more expensive option than the former, however, this can amplify the desired frequency while suppressing the noise, which makes the signal-to-noise ratio greater. Both options can be useful and serve our purpose of noise cancellation well.\par 
Additionally, data calibration against humidity and temperature could be achieved by integrating a humidity and temperature sensor into the circuit board and measuring their effect on the output voltage and then subtracting these elements from the raw data itself. As a result, the circuit would be more resistant against humidity and temperature changes in the environment and the outputted values would be even more dependent on the \chem{NO_2} concentration rather than other environmental effects.\par
Lastly, it is necessary to test the circuit under different environmental conditions with different \chem{NO_2} concentrations. The method used for examining the sensors were fine for testing purposes, although another experiment with varying \chem{NO_2} concentrations can certify the success of the outcome.



