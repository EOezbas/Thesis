\chapter{Outlook}
\label{sec:outlook}

\section{Possible Future Enhancements}
Directly measuring the output voltage gives us in theory the raw data of the \chem{NO_2} concentration, since the output voltage increases or decreases linearly as the ppb level of \chem{NO_2} changes. \textbf{grafik}, however the data will depend on electromagnetic noise and environmental elements -especially from humidity and temperature- and thus be somewhat inconsistent. The raw data obtained directly from the circuit must be calibrated against such noises in order to get a better approximation of real concentration values. This can be achieved through altering the circuit and making it more resistent to electromagnetic noise. There are two possible ways of dealing with such noises: Passive and active filtering. The former is achieved by adding passive elements like resistors, capacitors and inductors. After detecting the peak frequency of the noise contained in the raw signal (possibly with the help of an oscilloscope by applying Fourier Transformation on the signal and acquiring the frequency spectrum of the signal) it can be eliminated by adding high and low-pass filters. The latter option requires an outside power source, hence the name "active", which makes it a more expensive option than the former, however this can amplify the desired frequency while surpressing the noise, which makes the signal-to-noise ratio larger. Both options can be useful and serve our purpose of noise cancellation.\par 
Data calibration against humidity and temperature could be achieved by integrating a humidity and temperature sensor into the circuit board and measuring their effect on the output voltage and then subtracting these elements from the raw data itself.

%wifi

