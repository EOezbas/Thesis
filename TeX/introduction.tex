\chapter{Introduction}
\label{sec:introduction}


As the world population increases by millions every year, the environmental damage we cause increases dramatically. In densely populated areas -and especially in larger cities- air pollution is a major problem, which does not only have financial consequences but also affects the quality of our lives in many ways. Since air pollution today constitutes a significant problem, there are more than many researches and studies regarding this issue: " . . . the total damage costs of air pollution [is estimated] to be US\$ 3.0 trillion in 2010, or 5.6\% of Gross World Product (GWP). These losses are equivalent to US\$ 430 for every person on the planet."\cite{Hutton2011} is from just one of the numerous studies made on financial damage caused by air pollution.\par
In order to assess this problem correctly and take suitable measures to minimize the harm of air pollution, one should first be capable of finding out the cause accurately. Only after an accurate diagnosis can there be a suitable solution and thus a significant outcome. When it comes to air pollution, one of the best ways to detect the cause is to make density measurements of air pollutants in various locations with electrochemical sensors sensitive to specific gases. However there are some requirements that must be fulfilled: "To adequately characterize air quality (AQ), measurements must be fast (real-time), scalable, and reliable (with known accuracy, precision, and stability over time)."\cite{Cross2017} The more accurate and fast the sensors get, the more expensive the gas measurement station will be. Since it is important to make measurements in multiple locations to create a pool of air pollutant density data and thus getting a better understanding of the environmental damage, a collective of stations are needed, which increases the total cost dramatically. For this reason, a low-cost gas measurement station could be a simpler and cheaper solution.\par
The goal of this bachelor thesis is to design and build a low-cost circuit suited for Alphasense NO2-B43F sensor, which is sensitive to \chem{NO_2} concentration in the air. This circuit is capable of receiving real-time information concerning \chem{NO_2} concentration and storing the data into an SD-Card. In addition to that documenting the steps well and describing the final project in detail also plays a very important part, since it enables to create the same sensor station more efficiently, which means with less time and resources spent. \par 
In this writing researches on means of measuring \chem{NO_2} concentrations in different locations are reviewed and possible structures of gas measurement stations are examined. The results are then shown and the challenges of designing and building \chem{NO_2} measurement stations are explained. Lastly, some possible future enhancements to the latest design are discussed and the documents considering the hardware design as well as the source code are provided. 
