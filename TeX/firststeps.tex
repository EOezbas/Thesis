\chapter{Preparation}
\label{sec:firststeps}

\section{Initial Researches}
Some research on the working principle of Alphasense \chem{NO_2} electrochemical sensor had to be made prior to the initialization of the mechanical and electronical sections of this project. The paper /////// on previous experiments conducted in Boston, United States of America was very helpful to get a first idea about how I could start building my project designed for \chem{NO_2} density measurements. In another paper about a research conducted in Zurich, Switzerland    Previously mentioned researchs did not provide much direct information about the circuit itself and were realistically not essential for me to realize my project, as my goal was to build a functioning circuit, to get meaningful data from it and to document the entire process of my project well. However, the researchs and the papers I read gave me even more motivation about how topical air pollution is and thus how important it is to try to build a low-cost air pollutant density measuring station and to collect useful environmental data.

\section{Getting a Better Understanding}
I had to read the application notes on the Alphasense Webpage to get a better understanding of the inner structure as well as the pinout of the sensor and for this purpose I began to study the /////////////// (kaynak) (inner structure). As I started to understand which electrode of the sensor was responsible for which purpose, I began to get an idea of how I could build my own circuit, which would be able to supply enough current to the sensor and output voltage, linearly proportional to the concentration of the air pollutant, in other words, the ppb level of \chem{NO_2} in the air. \par
Afterwards I started to read //////////////////(designing a circuit), which gave me a starting point for the circuit. In diagram ////// from //// you can see a circuit design for a three-electrode sensor. I was actaully working with the sensor NO2-B43F, which is a four-electrode sensor, but this circuit schematic was nonetheless a good point to start building and testing the circuit.

\section{}
I started to make a list of electronical parts to start constructing the circuit according to the preivously mentioned circuit schematic from Alphasense. The circuit consists 


