\chapter{Preparation}
\label{sec:firststeps}

\section{Initial Researches}
Some research on the working principle of Alphasense \chem{NO_2} electrochemical sensor had to be made prior to the initialization of the mechanical and electronical sections of this project. The paper /////// on previous experiments conducted in Boston, United States of America was very helpful to get a first idea about how I could start building my project designed for \chem{NO_2} density measurements. In another paper about a research conducted in Zurich, Switzerland, ///    Previously mentioned researchs did not provide much direct information about the circuit itself and were realistically not essential for me to realize my project, as my goal was to build a functioning circuit, to get meaningful data from it and to document the entire process of my project well. However, the researchs and the papers I read gave me even more motivation about how topical air pollution is and thus how important it is to try to build a low-cost air pollutant density measuring station and to collect useful environmental data.

\section{Getting a Better Understanding}
I had to read the application notes on the Alphasense Webpage to get a better understanding of the inner structure as well as the pinout of the sensor and for this purpose I began to study the /////////////// (kaynak) (inner structure). As I started to understand which electrode of the sensor was responsible for which purpose, I began to get an idea of how I could build my own circuit, which would be able to supply enough current to the sensor and output voltage, linearly proportional to the concentration of the air pollutant, which in other words is the ppb level of \chem{NO_2} in the air. \par
Afterwards I started to read //////////////////(designing a circuit), which gave me a starting point for the circuit. In diagram ////// from //// you can see a circuit design for a three-electrode sensor. I was actaully working with the sensor NO2-B43F, which is a four-electrode sensor, but this circuit schematic was nonetheless a good point to start building and testing the circuit.

\chapter{Putting Into Practice}

\section{Design of the Circuit}

In order to understand the circuit fully, we must first study the structure and the pinout of the NO2-B43F sensor. There are 4 electrodes of the electrochemical sensor: Working electrode, reference electrode, counter electrode and auxilliary electrode. The reference electrode holds the potential of the working electrode stable at a certain level, which is equal to the potential of the reference electrode itself. This potential must be fixed to ensure a stable outcome from the sensor and the circuit. The counter electrode must be able to supply enough current to the working electrode, so that the current through the counter and working electrode can be translated and amplified to create the output voltage.\par
The A4 and B4 sensors have two sensing electrodes: The working electrode and the auxiliary electrode. The main purpose of the working electrode is to react to the \chem{NO_2} in the air and thus create a current flow proportional to the gas concentration. In other words the working electrode responds to gas concentration whereas the auxiliary electrode does not respond to gas. The idea is to be able to correct for zero drifts using the auxiliary electrode output. It is thus recommended that at the beginning both electrode outputs are recorded (Working electrode and auxiliary electrode) rather than applying a correction directly.\par
The circuit in general comprises of operational amplifiers, a couple of capacitors to reduce noise from the sensor and various resistors to get the desired gain from the operational amplifiers.In Figure ////// the detailed circuit schematic is given. On the left hand side of the circuit the NO2-B43F sensor is connected via 3 electrodes to the circuit. This schematic is for 3 electrode sensors like previously mentioned, so the auxilliary electrode is left out. The power connections of the operational amplifiers are also not shown, however they must be provided a regulated and high enough voltage to encompass the maxima of the output voltage. In addition to that the power supply must be rated with high enough current to be able to supply the required current drawn by the counter electrode and the operational amplifiers themselves.\par
The circuit consists of 2 stages of amplifiers. The first stage is the control circuit, whose main objective is to supply the counter electrode with enough current so that the current required by the working electrode is counterpoised. The potential at the reference electrode, namely the reference voltage is connected to the inverting input of the operational anplifier

aux olmadığı için sıkıntı oldu.   

, which flows through the sensing resistor R\textsubscript{Load}, 




I started to make a list of electronical parts to start constructing the circuit according to the previously mentioned circuit schematic from Alphasense. 


\section{}
The circuit mainly consists of 3 operational amplifiers, a couple of capacitors to reduce noise from the power supply and various resistors to get the required output voltage to input voltage ratio from each operational amplifier. The circuit and the purpose of its components will be explained in detail in the following section. 


