\chapter{Challenges}
\label{sec:challenges}

\section{Converting SMD to THT}
After preparing an orders list according to the previously mentioned circuit design taken from Alphasense \cite{2009} and receiving these components, the first challenge becomes very clear: Transforming the surface-mount device (SMD) parts into through-hole components. The operational amplifiers used in the circuit (LT6011) are only available to order in SMD form. Since the SMD components are needed to verify the circuit design, one cannot simply design a printed circuit board and solder the components on the board without testing the circuit design with a breadboard first. Otherwise, everything becomes inalterable and the tiniest change in circuit schematic leads to a whole new circuit design and thus a printed circuit board from scratch. That is why it is recommended to build an adapter for making the SMD modules breadboard compatible first. \par 
For this purpose, a perfboard and male sockets are used. The male sockets are split into two 4-pin male sockets since the LT6011 ICs have an 8-pin structure, namely 4 pins on both sides. Afterward, the sockets are soldered parallel to each other with a reasonable distance in between so that there is enough room for the IC to fit between the two sockets. Here it is quite important to first solder the sockets on the side of the perfboard, on which the holes are surrounded by copper rings, with the longer part of the sockets laying on this side. The IC will be fixed on the other side of the perfboard because the distance between any two pins of the IC is much smaller than the distance between two holes of the perfboard and thus this leads to a short between the pins. Afterward, the IC is placed between the sockets and anchored on the perfboard by gluing with hot glue. This makes it easier to make the connections between the sockets and the IC itself since it stays fixed during soldering. It can be tricky to solder one pin of the IC with one pin of the socket since there is no such surface e.g copper between those two on which the solder can stick. This makes the soldering process much harder since the solder tends to stay on one the IC or the socket, thus not binding them together. It is therefore recommended to make the distance between the two sockets as small as possible so that the IC and the socket stay as close as possible. This makes the soldering much easier since the distance between the two parts is smaller and a connection between the two can be made with enough solder applied. Another option is to use tiny wires between the parts. This way the solder can stick onto this wire and make the connection instead of just creating a solder bubble on one pin. Here it is recommended to fix the wire by gluing it on the perfboard or use a third-hand soldering stand to stabilize the wire and the other parts.

\section{Initial Design}

\subsection{Inconsistencies Between Design and Sensor}
The circuit schematic taken from Alphasense webpage \cite{2009} illustrates a design for a 3-electrode sensor, but the goal is to build a circuit for the 4-electrode sensor NO2-B43F. This leads to some problems concerning the correctness of the output data. The three electrodes of the sensor in the schematic, namely the working, counter and reference electrodes, have the same purposes like the three electrodes of the NO2-B43F. However, according to this design, the auxiliary electrode is not used at all, which is the fourth electrode of our sensor. This electrode is used for noise cancellation and zero drift correction. Not using the auxiliary electrode leads to an output with high noise ratio, because firstly the electrode is floating (not connected to any stable potential) and secondly it can not be used for corrections in the output voltage. 

\subsubsection{Testing the Sensors}
Since there are not any containers of \chem{NO_2} in the lab, an alternative solution had to be found to test the response of both sensors (the one on the ISB circuit and the one connected to my circuit) to a change of \chem{NO_2} concentration in the air. For this reason, the sensors are fixed in close proximity and parallel to each other onto a portable board in order to have the same environmental conditions for both of the sensors. Afterward, the sensors are fanned with the sensing areas of the sensors facing directly to the fan. Normally gas flow parallel to the sensing area of the sensor must be supplied to the sensors, since the sensors are sensitive to air flow perpendicular to the sensing area which causes a change in the outputted signal although the \chem{NO_2} concentration in the air does not change. For this reason, the sensors are fanned in order to create a perpendicular gas flow and thus create a change in the output signals of both sensors, of course only for testing purposes. This way the reaction velocity and amplitude of the output signals can be compared.\par 
For the reasons explained in the previous section the outputted voltage values are different compared to the values gathered from the ISB circuit. The two circuit boards are tested in the same environment and for both circuits, the same power supply unit is used. For that reason, the values have to be close to each other. A possible solution to this problem is changing the software and adapting it to suppress the noise consisted of the output signal. For this reason, another function to the Arduino code is added in order to make the Arduino print the average voltage level over a specific time interval instead of printing the values directly. This results in better values with lower SNR, but the response velocity and amplitude are different compared to the values from the ISB circuit. Additionally, this is not a befitting solution since the values coming from the circuit are faulty and a correction in the hardware is needed.  



\section{Final Design}
\subsection{Critical parameters}
\label{sub:criticalParameters}

It is of utmost importance to pay attention to the resistor values in the circuit schematic and the resistors used in the actual circuit. A little difference in these values creates a large spread between the desired output voltage and the actual output since most of the resistors in this circuit design play a role in determining the gain of the operational amplifiers. An inaccuracy in resistor value changes the gain of the operational amplifier and consequently the output voltage gets multiplied with that error. If the resistor inaccuracy is in earlier stages of the circuit, the error at the output side gets even higher since the error gets amplified by the second operational amplifier in the last amplifying stage. It is recommended to check the resistors' values preferably with a multimeter before mounting it on the circuit, even if the circuit is being built on a breadboard and will be modifiable, since there are a lot of resistors in this circuit design and it can be quite complicated to detect where the problem lies after the circuit is completed. For this reason, it is quite important to check every part individually before putting them into the circuit. \par 

\subsubsection{Inconsistencies Between Design and Board}
The results from the circuit are higher than the output values of the ISB circuit. For this reason, a constant number from the voltage values are subtracted. As the outputs are stable, meaning that the sensors are not fanned and the test is conducted indoors with little to no change in the \chem{NO_2} concentration in the air, the values are approximately equal to each other. However, after fanning the sensors, the difference in the response amplitude and response velocity is noticeable. Therefore a comparison between the board and the schematic is needed. After comparing the circuit design in Figure  ~\ref{fig:A.2} and the ISB board, some discrepancies between the two becomes visible. Some of the resistors' values are different from each other. The values of the resistors $R29$ and $R31$ are depicted as 4.7\,kilohms but the resistors on the ISB board are actually 1.5\,kilohm resistors. This leads to a different input voltage at the non-inverting inputs of the ICs, which causes a different result outputted from the circuit. After correcting these values and editing out the parts which are not used in this project, the circuit schematic in Figure  ~\ref{fig:A.3} is created. With the help of this edited design, the values explained in $Results$ section are reached.