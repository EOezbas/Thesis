\chapter{Challenges}
\label{sec:challenges}

\section{Converting SMD to THT}
After preparing an orders list according to the previously mentioned circuit design taken from Alphasense \cite{2009} and receiving these components, my first challenge was very clear: Transforming the surface-mount device (SMD) parts into through-hole components. The operational amplifiers used in the circuit (LT6011) were only available to order in SMD form. Since I needed the SMD components to verify the circuit design, I could not simply design a printed circuit board and solder the components on the board without testing the circuit design with a breadboard first. Otherwise, everything would be inalterable and the tiniest change in circuit schematic would lead to a whole new circuit design and thus a printed circuit board from scratch. That is why I started to build an adapter for making the SMD modules breadboard compatible. \par 
For this purpose, I used a perfboard and male sockets. I first split the male sockets into two 4-pin male sockets, since the LT6011 ICs have an 8-pin structure, namely 4 pins on both sides. Afterward, I soldered these sockets parallel to each other with a reasonable distance in between so that there is enough room for the IC to fit between the two sockets. Here it is quite important to first solder the sockets on the side of the perfboard with copper rings around the holes with the longer part of the sockets laying on this side. The IC will be fixed on the other side of the perfboard because the distance between any two pins of the IC is much smaller than the distance between two holes of the perfboard and this would thus lead to a short between the pins. Afterward, I placed the IC between the sockets and anchored it on the perfboard by gluing it with hot glue. This makes it easier to make the connections between the sockets and the IC itself since it stays fixed during soldering. It can be tricky to solder one pin of the IC with one pin of the socket since there is no such surface e.g copper between those two on which the solder can stick. This makes the soldering process much harder since the solder tends to stay on one the IC or the socket, thus not binding them together. It is therefore recommended to make the distance between the two sockets as small as possible so that the IC and the socket stay as close as possible. This makes the soldering much easier since the distance between the two parts is smaller and a connection between the two can be made with enough solder applied. Another option would be to use tiny wires between the parts. This way the solder can stick onto this wire and make the connection instead of just creating a solder bubble on one pin. Here it is recommended to fix the wire by gluing it on the perfboard or use a third-hand soldering stand to stabilize the wire and the other parts.

\section{Initial Design}

\subsection{Inconsistencies Between Design and Sensor}
The circuit schematic taken from Alphasense webpage \cite{2009} illustrates a design for a 3-electrode sensor, but what we wanted to build was a circuit for the 4-electrode sensor NO2-B43F. This led to some problems concerning the correctness of the output data. The three electrodes of the sensor in the schematic, namely the working, counter and reference electrodes, have the same purposes like the three electrodes of the NO2-B43F. However, according to this design, the auxiliary electrode was not used at all, which is the fourth electrode of our sensor. This electrode is used for noise cancellation and zero drift correction. Not using the auxiliary electrode led to an output with high noise ratio, because firstly the electrode was floating (not connected to any stable potential) and secondly it could not be used for corrections in the output voltage. 

\subsubsection{Testing the Sensors}
Since we did not have any containers of \chem{NO_2} in the lab, I had to find another way to test the response of both sensors (the one on the ISB circuit and the one connected to my circuit) to a change of \chem{NO_2} concentration in the air. So I fixed the sensors in close proximity and parallel to each other onto a portable board in order to have the same environmental conditions for both of the sensors. Afterward, I fanned the sensors with the sensing areas of the sensors facing directly to the fan. Normally gas flow parallel to the sensing area of the sensor must be supplied to the sensors, since the sensors are sensitive to air flow perpendicular to the sensing area which causes a change in the outputted signal although the \chem{NO_2} concentration in the air does not change. For this reason, the sensors were fanned in order to create a perpendicular gas flow and thus create a change in the output signals of both sensors, of course only for testing purposes. This way I could compare the reaction velocity and amplitude of the output signals in millivolts.\par 
For the reasons explained in the previous section the outputted voltage values were different compared to the values gathered from the ISB circuit. The two circuit boards were tested in the same environment and for both circuits, the same power supply unit was used. For that reason, the values should have been close to each other. I tried to solve this problem by changing the software and adapting it to suppress the noise consisted of the output signal. For this reason, I added another function to the Arduino code in order to make the Arduino print the average voltage level over a specific time interval instead of printing the values directly. This resulted in better values with lower SNR, but the response velocity and amplitude were different compared to the values from the ISB circuit. Additionally, this was not a befitting solution since the values coming from my circuit were faulty and a correction in the hardware was needed.  



\section{Final Design}
\subsection{Critical parameters}
\label{sub:criticalParameters}

It is of utmost importance to pay attention to the resistor values in the circuit schematic and the resistors used in the actual circuit. A little difference in these values creates a large spread between the desired output voltage and the actual output since most of the resistors in this circuit design play a role in determining the gain of the operational amplifiers. An inaccuracy in resistor value changes the gain of the operational amplifier and consequently the output voltage gets multiplied with that error. If the resistor inaccuracy is in earlier stages of the circuit, the error at the output side gets even higher since the error gets amplified by the second operational amplifier in the last amplifying stage. It is recommended to check the resistors' values preferably with a multimeter before mounting it on the circuit, even if the circuit is being built on a breadboard and will be modifiable, since there are a lot of resistors in this circuit design and it can be quite complicated to detect where the problem lies after the circuit is completed. For this reason, it is quite important to check every part individually before putting them into the circuit. \par 

\subsubsection{Inconsistencies Between Design and Board}
The results from my circuit were higher than the output values of the ISB circuit. For this reason, I firstly subtracted a constant number from the voltage values. As the outputs were stable, meaning that the sensors were not fanned and the test was conducted indoors with little to no change in the \chem{NO_2} concentration in the air, the values were approximately equal to each other. However after fanning the sensors the difference in the response amplitude and response velocity was noticeable. Therefore I started to compare the board and the schematic. After comparing the circuit design in Figure  ~\ref{fig:A.2} and the ISB board, I detected some discrepancies between the two. I found out that some of the resistors' values were different from each other. The values of the resistors $R29$ and $R31$ were depicted as 4.7 kilohms but the resistors on the ISB board were actually 1.5 kilohm resistors. This led to a different input voltage at the non-inverting inputs of the ICs, which caused a different result outputted from the circuit. After correcting these values and editing out the parts which were not used in this project, I came up with the circuit schematic in Figure  ~\ref{fig:A.3}. With this edited design I was able to reach the values explained in $Results$ section.