\chapter{Challenges}
\label{sec:challenges}


\section{Initial Design}
\subsection{Mechanical and optical challenges}
\label{sub:mechanicalChallenges}

After preparing an orders list according to the previously mentioned circuit design taken from Alphasense \textbf{dokuman} and receiving these components, my first challenge was very clear: Transforming the surface-mount device (SMD) parts into through-hole components. The operational amplifiers used in the circuit (LT6011) were only available to order in SMD form. Since I needed the SMD components to verify the circuit design, I could not simply design a printed circuit board and solder the components on the board without testing the circuit design with a breadboard first. Otherwise everything would be inalterable and the tiniest change in circuit schematic would lead to a whole new circuit design and thus a printed circuit board from scratch. That is why I started to build an adapter for making the SMD modules breadboard compatible. \par 
For this purpose I used a perfboard and male sockets. I first split the male sockets into two 4-pin male sockets, since the LT6011 ICs have an 8-pin structure, namely 4 pins on both sides. Afterwards I soldered these sockets parallel to each other with reasonable distance in between so that there is enough room for the IC to fit between the two sockets. Here it is quite important to first solder the sockets on the side of the perfboard with copper rings around the holes with the longer part of the sockets laying on this side. The IC will be fixed on the other side of the perfboard because the distance between any two pins of the IC is much smaller than the distance between two holes of the perfboard and this would thus lead to a short between the pins. Afterwards I placed the IC between the sockets and anchored it on the perfboard by gluing it with hot glue. This makes it easier to make the connections between the sockets and the IC itself, since it stays fixed during soldering. It can be tricky to solder one pin of the IC with one pin of the socket since there is no such surface e.g copper between those two on which the solder can stick. This makes the soldering process much harder since the solder tends to stay on one the IC or the socket, thus not binding them together. It is therefore recommended to make the distance between the two sockets as small as possible, so that the IC and the socket stays as close as possible. This makes the soldering much easier since the distance between the two parts is smaller and a connection between the two can be made with enough solder applied. Another option would be to use tiny wires between the parts. This way the solder can stick onto this wire and make the connection instead of just creating a solder bubble on one pin. Here it is recommended to fix the wire by gluing it on the perfboard or use a third hand soldering stand to stabilize the wire and the other parts.     

\section{Final Design}
\subsection{Mechanical and optical challenges}


\subsection{Critical parameters}
\label{sub:criticalParameters}

It is of utmost importance to pay attention to the resistor values in the circuit schematic and the resistors used in the actual circuit. A little difference in these values creates a large spread between the desired output voltage and the actual output, since most of the resistors in this circuit design plays a role in determining the gain of the operational amplifiers. An inaccuracy in resistor value changes the gain of the operational amplifier and consequently the output voltage gets multiplied with that error. If the resistor inaccuracy is in earlier stages of the circuit, the error at the output side gets even higher since the error gets amplified by the second operational amplifier in the last amplifying stage. It is recommended to check the resistors' values preferably with a multimeter before mounting it on the circuit, even if the circuit is being built on a breadboard. There are a lot of resistors in this circuit design and it can be really complicated to detect where the problem lies after the circuit is completed. For this reason it is quite important to check every part individually before putting them into the circuit. \par 
In my case I built the circuit exatly how it was depicted on the circuit schematic and       